\chapter{Introduction and Goals} 

Hello! Allow me to introduce myself. My name is Niels, and by the time you read this, I will hopefully have a Ph.D from Stanford's Computer Science department. As of this writing, I'm in my 6th year, and working towards a thesis focusing on automating the use of Quadrotor Aircraft equipped with Cameras to perform cinematography! 

This collection of notes is my attempt to build a structure of sorts in which I can place the knowledge I'm gaining as I explore these worlds. You can view them as an account of my wanderings into the High Country of the Mind\cite{Pirsig2005}, exploring (mostly) old and (hopefully) a few new paths relating to the blossoming field of micro aerial vehicles and their usage in Computer Graphics.  
 \footnote{Shamelessly stealing metaphors from Robert Pirsig's Zen and the Art of Motorcycle Maintenance. One of my favorite books, both for a love of long distance motorcycle riding I share with the author, and it's views on the important strangeness of Quality.}

Now, before we jump into a big endeavour such as writing or reading a selection of notes, it's important that we establish some goals for this document. I find that, with good measurable goals publically stated, I work much better and my colleagues and friends find it easier to collaborate with me. Hopefully the same effect will extend here!

With this document I'd like to:

\begin{enumerate}

\item Create a continuity of ideas over the multiple years of work and thought in this field, and thus avoid losing too much of what myself and my team learns and thinks to the rush of getting on to the Next Thing.

\item Learn through Writing. Writing concepts down, just like teaching, forces me to clarify them and patch holes in my understanding! \footnote{``Writing is nature's way of letting you know how sloppy your thinking is.'' - Guindon (cartoon)

``You have to learn, so you know what is it you know and don't know. In Science you must be very careful not to confuse yourself, which is very easy to do!'' - Richard Feynman.} 

\item Disseminate what I've learned to my colleagues, especially as interest in this area grows

\item Track my own progress, what I don't know yet and what I've figured out, in a place where I can see the progress as I struggle and fight my way up into these mountains.\footnote{Keeping up the motivation as a graduate student by tracking progress is crucial!}

\item Record some of the things that didn't work, the trails that led to a dead end. There's no place to publish such stuff, yet 

\end{enumerate}