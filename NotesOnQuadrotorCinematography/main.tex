\documentclass{tufte-book}

\hypersetup{colorlinks}% uncomment this line if you prefer colored hyperlinks (e.g., for onscreen viewing)

%%
% Book metadata
\title{Notes on \\Quadrotor \\Cinematography\thanks{}}
\author[Niels Joubert niels@cs.stanford.edu]{Niels Joubert}
\publisher{http://njoubert.com/}

%%
% If they're installed, use Bergamo and Chantilly from www.fontsite.com.
% They're clones of Bembo and Gill Sans, respectively.
%\IfFileExists{bergamo.sty}{\usepackage[osf]{bergamo}}{}% Bembo
%\IfFileExists{chantill.sty}{\usepackage{chantill}}{}% Gill Sans

%\usepackage{microtype}

%%
% Just some sample text
\usepackage{lipsum}

%%
% For nicely typeset tabular material
\usepackage{booktabs}

%%
% For graphics / images
\usepackage{graphicx}
\setkeys{Gin}{width=\linewidth,totalheight=\textheight,keepaspectratio}
\graphicspath{{graphics/}}

% The fancyvrb package lets us customize the formatting of verbatim
% environments.  We use a slightly smaller font.
\usepackage{fancyvrb}
\fvset{fontsize=\normalsize}

%%
% Prints argument within hanging parentheses (i.e., parentheses that take
% up no horizontal space).  Useful in tabular environments.
\newcommand{\hangp}[1]{\makebox[0pt][r]{(}#1\makebox[0pt][l]{)}}

%%
% Prints an asterisk that takes up no horizontal space.
% Useful in tabular environments.
\newcommand{\hangstar}{\makebox[0pt][l]{*}}

%%
% Prints a trailing space in a smart way.
\usepackage{xspace}


\newcommand{\TL}{Tufte-\LaTeX\xspace}

% Prints the month name (e.g., January) and the year (e.g., 2008)
\newcommand{\monthyear}{%
  \ifcase\month\or January\or February\or March\or April\or May\or June\or
  July\or August\or September\or October\or November\or
  December\fi\space\number\year
}


% Prints an epigraph and speaker in sans serif, all-caps type.
\newcommand{\openepigraph}[2]{%
  %\sffamily\fontsize{14}{16}\selectfont
  \begin{fullwidth}
  \sffamily\large
  \begin{doublespace}
  \noindent\allcaps{#1}\\% epigraph
  \noindent\allcaps{#2}% author
  \end{doublespace}
  \end{fullwidth}
}

% Inserts a blank page
\newcommand{\blankpage}{\newpage\hbox{}\thispagestyle{empty}\newpage}

\usepackage{units}

% Typesets the font size, leading, and measure in the form of 10/12x26 pc.
\newcommand{\measure}[3]{#1/#2$\times$\unit[#3]{pc}}

% Macros for typesetting the documentation
\newcommand{\hlred}[1]{\textcolor{Maroon}{#1}}% prints in red
\newcommand{\hangleft}[1]{\makebox[0pt][r]{#1}}
\newcommand{\hairsp}{\hspace{1pt}}% hair space
\newcommand{\hquad}{\hskip0.5em\relax}% half quad space
\newcommand{\TODO}{\textcolor{red}{\bf TODO!}\xspace}
\newcommand{\ie}{\textit{i.\hairsp{}e.}\xspace}
\newcommand{\eg}{\textit{e.\hairsp{}g.}\xspace}
\newcommand{\na}{\quad--}% used in tables for N/A cells
\providecommand{\XeLaTeX}{X\lower.5ex\hbox{\kern-0.15em\reflectbox{E}}\kern-0.1em\LaTeX}
\newcommand{\tXeLaTeX}{\XeLaTeX\index{XeLaTeX@\protect\XeLaTeX}}
% \index{\texttt{\textbackslash xyz}@\hangleft{\texttt{\textbackslash}}\texttt{xyz}}
\newcommand{\tuftebs}{\symbol{'134}}% a backslash in tt type in OT1/T1
\newcommand{\doccmdnoindex}[2][]{\texttt{\tuftebs#2}}% command name -- adds backslash automatically (and doesn't add cmd to the index)
\newcommand{\doccmddef}[2][]{%
  \hlred{\texttt{\tuftebs#2}}\label{cmd:#2}%
  \ifthenelse{\isempty{#1}}%
    {% add the command to the index
      \index{#2 command@\protect\hangleft{\texttt{\tuftebs}}\texttt{#2}}% command name
    }%
    {% add the command and package to the index
      \index{#2 command@\protect\hangleft{\texttt{\tuftebs}}\texttt{#2} (\texttt{#1} package)}% command name
      \index{#1 package@\texttt{#1} package}\index{packages!#1@\texttt{#1}}% package name
    }%
}% command name -- adds backslash automatically
\newcommand{\doccmd}[2][]{%
  \texttt{\tuftebs#2}%
  \ifthenelse{\isempty{#1}}%
    {% add the command to the index
      \index{#2 command@\protect\hangleft{\texttt{\tuftebs}}\texttt{#2}}% command name
    }%
    {% add the command and package to the index
      \index{#2 command@\protect\hangleft{\texttt{\tuftebs}}\texttt{#2} (\texttt{#1} package)}% command name
      \index{#1 package@\texttt{#1} package}\index{packages!#1@\texttt{#1}}% package name
    }%
}% command name -- adds backslash automatically
\newcommand{\docopt}[1]{\ensuremath{\langle}\textrm{\textit{#1}}\ensuremath{\rangle}}% optional command argument
\newcommand{\docarg}[1]{\textrm{\textit{#1}}}% (required) command argument
\newenvironment{docspec}{\begin{quotation}\ttfamily\parskip0pt\parindent0pt\ignorespaces}{\end{quotation}}% command specification environment
\newcommand{\docenv}[1]{\texttt{#1}\index{#1 environment@\texttt{#1} environment}\index{environments!#1@\texttt{#1}}}% environment name
\newcommand{\docenvdef}[1]{\hlred{\texttt{#1}}\label{env:#1}\index{#1 environment@\texttt{#1} environment}\index{environments!#1@\texttt{#1}}}% environment name
\newcommand{\docpkg}[1]{\texttt{#1}\index{#1 package@\texttt{#1} package}\index{packages!#1@\texttt{#1}}}% package name
\newcommand{\doccls}[1]{\texttt{#1}}% document class name
\newcommand{\docclsopt}[1]{\texttt{#1}\index{#1 class option@\texttt{#1} class option}\index{class options!#1@\texttt{#1}}}% document class option name
\newcommand{\docclsoptdef}[1]{\hlred{\texttt{#1}}\label{clsopt:#1}\index{#1 class option@\texttt{#1} class option}\index{class options!#1@\texttt{#1}}}% document class option name defined
\newcommand{\docmsg}[2]{\bigskip\begin{fullwidth}\noindent\ttfamily#1\end{fullwidth}\medskip\par\noindent#2}
\newcommand{\docfilehook}[2]{\texttt{#1}\index{file hooks!#2}\index{#1@\texttt{#1}}}
\newcommand{\doccounter}[1]{\texttt{#1}\index{#1 counter@\texttt{#1} counter}}

% Generates the index
\usepackage{makeidx}
\makeindex

\begin{document}

\frontmatter

\maketitle

% v.2 epigraphs
\newpage\thispagestyle{empty}
\openepigraph{%
The art of the subject... by which we meant the kind of mastery that comes from an intimate familiarity with real circuits, actual devices and the like
}{Paul Horowitz and Winfield Hill, {\itshape The Art of Electronics}
}
\vfill

% v.4 copyright page
\newpage
\begin{fullwidth}
~\vfill
\thispagestyle{empty}
\setlength{\parindent}{0pt}
\setlength{\parskip}{\baselineskip}
Copyright \copyright\ \the\year\ \thanklessauthor

\par\smallcaps{Published by \thanklesspublisher}

\par\smallcaps{tufte-latex.github.io/tufte-latex/}

\par Licensed under the Apache License, Version 2.0 (the ``License''); you may not
use this file except in compliance with the License. You may obtain a copy
of the License at \url{http://www.apache.org/licenses/LICENSE-2.0}. Unless
required by applicable law or agreed to in writing, software distributed
under the License is distributed on an \smallcaps{``AS IS'' BASIS, WITHOUT
WARRANTIES OR CONDITIONS OF ANY KIND}, either express or implied. See the
License for the specific language governing permissions and limitations
under the License.\index{license}

\par\textit{First printing, \monthyear}
\end{fullwidth}

% r.5 contents
\tableofcontents

% \listoffigures

% \listoftables

% r.9 introduction
%\cleardoublepage


%%
% Start the main matter (normal chapters)
\mainmatter

\chapter{Introduction and Goals} 

Hello! Allow me to introduce myself. My name is Niels, and by the time you read this, I will hopefully have a Ph.D from Stanford's Computer Science department. At the start of writing this document, I'm in my 6th year, and working towards a thesis focusing on automating the use of quadrotor aircraft equipped with cameras to perform cinematography! 

This collection of notes is my attempt to build a structure of sorts in which I can place the knowledge I'm gaining as I explore these worlds. You can view them as an account of my wanderings into the High Country of the Mind\cite{Pirsig2005}, exploring (mostly) old and (hopefully) a few new paths relating to the blossoming field of micro aerial vehicles and their usage in Computer Graphics.  
 \footnote{Shamelessly stealing metaphors from Robert Pirsig's Zen and the Art of Motorcycle Maintenance. One of my favorite books, both for a love of long distance motorcycle riding I share with the author, and it's views on the important strangeness of Quality.}

Now, before we jump into a big endeavor such as writing or reading a selection of notes, it's important that we establish some goals for this document. I find that, with good measurable goals publicly stated, I work much better and my colleagues and friends find it easier to collaborate with me. Hopefully the same effect will extend here!

With this document I'd like to:

\begin{enumerate}

\item Create a continuity of ideas over the multiple years of work and thought in this field, and thus avoid losing too much of what myself and my team learns and thinks to the rush of getting on to the Next Thing.

\item Learn through Writing. Writing concepts down, just like teaching, forces me to clarify them and patch holes in my understanding! \footnote{``Writing is nature's way of letting you know how sloppy your thinking is.'' - Guindon (cartoon)

``You have to learn, so you know what is it you know and don't know. In Science you must be very careful not to confuse yourself, which is very easy to do!'' - Richard Feynman.} 

\item Disseminate what I've learned to my colleagues, especially as interest in this area grows.

\item Track my own progress, what I don't know yet and what I've figured out, in a place where I can see the progress as I struggle and fight my way up into these mountains.\footnote{Keeping up the motivation as a graduate student by tracking progress is crucial!}

\item Record some of the things that didn't work, the trails that led to a dead end. There's no place to publish such stuff, yet 

\end{enumerate}

\chapter{Preface: Doing Research}

Research in Computer Science is a challening and potentially rewarding blend of:
\begin{itemize}
\item project management
\item reading and learning
\item thinking
\item problem-solving
\item experimentation
\item mathematical modeling
\item engineering systems
\item psychology studies
\item writing
\item public discourse in the form of public speaking and argueing
\item people management and collaboration skill
\end{itemize}
Becoming uncommonly good in all of these areas will serve you well during and after your Ph.D studies.

\section{Niels' Good Meeting Tips}

\begin{itemize}

\item[Agenda] A meeting needs an agenda. 

\end{itemize}

\section{Making Talks}
Talks are important, and fun! I refer you to the fantastic ``''

\section{Project Management Ideas}

\section{Maintaining Mental Health}

\subsection{Emotional Intelligence}

word up
indeed

\section{Research Traits to Develop}

- Integrity

- Courage



\chapter{Questions I'd Like To Know The Answer To}

There are many areas, both explored and new, that I'd like to understand thoroughly. It seems to me like the best way to do that is through asking questions. Questions are both the end-product and raw fuel of curiosity, so I'm yotting the questions I do not have a satisfying answer to here.

This section is truly intended for myself, and many a reader might find themselves either bored or stunned by my apparent ignorance. For that I apologize, I am in no way knowleadgable about everything I need to be, even less what I want to be. If you have an intuitive answer to these, please to email me a niels@cs.stanford.edu

\subsection{}

\subsection{Vision and Graphics}

\begin{enumerate}

\item How do we find correspondences between two sets of feature vectors?

\item How do we measure the difference and structure between two vector fields?

\end{enumerate}



\chapter{Recurring Themes and Techniques}

\section{Finding Correspondences}

Solved with Machine Learning (Set of training examples) (Bricolage)

Solves with variational approach e.g. framed as energy minimization (SIFT Flow)

\section{Maintaining Constraints Under Editing}

Differential Manipulation

\chapter{Project Ideas}

\section{Deeply-Connected Filtering for an array of range-based position sensors}

\section{Automatic(Semi?) Editing of Aerial Cinematography}

\subsection{Sub Problems}

alignment of data (multimodal feature matching)

finding useful points in footage

matching to a soundtrack

editing UI

\subsection{Cool examples}
Auto-crash
Auto-hangtime

\section{Auto-Transfer of shots from one environment to another}

find correspondences in the geometry between two scenes, then re-purpose a shot from the previous environment to the new one.

\section{Automatic Perching for Autonomous Quads while Filming}

Observation: People enter a different mode of behavior once the quad is fully autonomous and they're using Horus. For example, they'd like to get a live preview halfway-along the spline, then go and iterate on editing in the virtual environment for a bit. Or, they'd like to have the quad wait for a specific moment when they can film an event, but they're not sure when it'll start.

\footnote{ASIDE: What is the NAME of this phenomenon of people going into different modes? Jeff Heer: ``Effect of Latency on user behavior'': ``The Effects of Interactive Latency on Exploratory Visual Analysis'', ``The effect of System Response Time on interactive computer aided problem solving
''}

Hypothesis: We can build a system that finds good places for the quad to perch in the real world. ``Good'' can be defined at someplace where you can a) still see the action you wanted to see and b) can get as quickly as possible to the start of your spline. 

Some ideas: This can tie into Manolis' work on finding affordances. We want to find all the places that affords perching!

\section{Front-mounted gimbals for infinite pan/tilt/roll}

Put a gimbal on the front of an H-quad. Stabilize in 3 axis. Now we can do crazy stuff! The difficulty seems to be yaw stabilization, which is harder now, and balance, because you're fighting gravity. 

Downside: can't keep flying in one orientation of the quad and yaw the camera 360, would have to yaw the quad.

\include{./chapters/40000_QuadrotorDetails/quadtotor_modeling}

\chapter{Dead Kittens Pile}



\chapter{Appendix Z: Unrelated yet Inspiring Works}

I've been fortunate during my graduate school career to have explored several fields I found fascinating and fun. These offer little insight into Quadrotor Cinematography, but they did influence my thinking and I feel compelled to yot them down somewhere. Here they are:


\section{Computer Science}

\begin{itemize}

\item[Structure and Interpretation of Computer Programs] Abel and Sussman's legendary text on the fundamentals of Computer Programs. This was the first real computer science text I read as part of Berkeley's CS61A course taught by Prof. Brian Harvey. Getting through this was a monumental experience of having my entire view of the world changed on a scale I've never experienced from a book before or since. I always find peace and a mischiveous grin in the opening dedication whenever I get disillusioned with my profession. \footnote{This book is dedicated, in respect and admiration, to the spirit that lives in the computer.

``I think that it's extraordinarily important that we in computer science keep fun in computing. When it started out, it was an awful lot of fun. Of course, the paying customers got shafted every now and then, and after a while we began to take their complaints seriously. We began to feel as if we really were responsible for the successful, error-free perfect use of these machines. I don't think we are. I think we're responsible for stretching them, setting them off in new directions, and keeping fun in the house. I hope the field of computer science never loses its sense of fun. Above all, I hope we don't become missionaries. Don't feel as if you're Bible salesmen. The world has too many of those already. What you know about computing other people will learn. Don't feel as if the key to successful computing is only in your hands. What's in your hands, I think and hope, is intelligence: the ability to see the machine as more than when you were first led up to it, that you can make it more.'' -Alan J. Perlis (April 1, 1922-February 7, 1990)
}

\end{itemize}

\section{Graphic Design, Visual Design, Typography}

\begin{itemize}

\item[Thinking with Type]

\item[Tufte's Works] including 

\end{itemize}

\section{History of Silicon Valley and Computer Science}

\begin{itemize}

\item[The New New Thing] Silicon Graphics is particularly dear to me heart. My introduction to SGI machines at a young age set my on the path of computer science, made me the nerd I am today, and arguably defined the direction of my life.

\item[What The Dormouse Said] This tells the story of how a military instrument of the 40s and 50s became a symbol of personal freedom and counterculture thinking. And of course the exploration of a link between Psychedelics and Computer Science is just awesome!

\end{itemize}



%%
% The back matter contains appendices, bibliographies, indices, glossaries, etc.

\backmatter

\bibliography{main}
\bibliographystyle{plainnat}


\printindex

\end{document}

