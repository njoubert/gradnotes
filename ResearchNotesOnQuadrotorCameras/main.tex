\documentclass{tufte-book}

\hypersetup{colorlinks}% uncomment this line if you prefer colored hyperlinks (e.g., for onscreen viewing)

%%
% Book metadata
\title{Notes on \\Quadrotor \\Cinematography\thanks{}}
\author[Niels Joubert niels@cs.stanford.edu]{Niels Joubert}
\publisher{http://njoubert.com/}

%%
% If they're installed, use Bergamo and Chantilly from www.fontsite.com.
% They're clones of Bembo and Gill Sans, respectively.
%\IfFileExists{bergamo.sty}{\usepackage[osf]{bergamo}}{}% Bembo
%\IfFileExists{chantill.sty}{\usepackage{chantill}}{}% Gill Sans

%\usepackage{microtype}

%%
% Just some sample text
\usepackage{lipsum}

%%
% For nicely typeset tabular material
\usepackage{booktabs}

%%
% For graphics / images
\usepackage{graphicx}
\setkeys{Gin}{width=\linewidth,totalheight=\textheight,keepaspectratio}
\graphicspath{{graphics/}}

% The fancyvrb package lets us customize the formatting of verbatim
% environments.  We use a slightly smaller font.
\usepackage{fancyvrb}
\fvset{fontsize=\normalsize}

%%
% Prints argument within hanging parentheses (i.e., parentheses that take
% up no horizontal space).  Useful in tabular environments.
\newcommand{\hangp}[1]{\makebox[0pt][r]{(}#1\makebox[0pt][l]{)}}

%%
% Prints an asterisk that takes up no horizontal space.
% Useful in tabular environments.
\newcommand{\hangstar}{\makebox[0pt][l]{*}}

%%
% Prints a trailing space in a smart way.
\usepackage{xspace}


\newcommand{\TL}{Tufte-\LaTeX\xspace}

% Prints the month name (e.g., January) and the year (e.g., 2008)
\newcommand{\monthyear}{%
  \ifcase\month\or January\or February\or March\or April\or May\or June\or
  July\or August\or September\or October\or November\or
  December\fi\space\number\year
}


% Prints an epigraph and speaker in sans serif, all-caps type.
\newcommand{\openepigraph}[2]{%
  %\sffamily\fontsize{14}{16}\selectfont
  \begin{fullwidth}
  \sffamily\large
  \begin{doublespace}
  \noindent\allcaps{#1}\\% epigraph
  \noindent\allcaps{#2}% author
  \end{doublespace}
  \end{fullwidth}
}

% Inserts a blank page
\newcommand{\blankpage}{\newpage\hbox{}\thispagestyle{empty}\newpage}

\usepackage{units}

% Typesets the font size, leading, and measure in the form of 10/12x26 pc.
\newcommand{\measure}[3]{#1/#2$\times$\unit[#3]{pc}}

% Macros for typesetting the documentation
\newcommand{\hlred}[1]{\textcolor{Maroon}{#1}}% prints in red
\newcommand{\hangleft}[1]{\makebox[0pt][r]{#1}}
\newcommand{\hairsp}{\hspace{1pt}}% hair space
\newcommand{\hquad}{\hskip0.5em\relax}% half quad space
\newcommand{\TODO}{\textcolor{red}{\bf TODO!}\xspace}
\newcommand{\ie}{\textit{i.\hairsp{}e.}\xspace}
\newcommand{\eg}{\textit{e.\hairsp{}g.}\xspace}
\newcommand{\na}{\quad--}% used in tables for N/A cells
\providecommand{\XeLaTeX}{X\lower.5ex\hbox{\kern-0.15em\reflectbox{E}}\kern-0.1em\LaTeX}
\newcommand{\tXeLaTeX}{\XeLaTeX\index{XeLaTeX@\protect\XeLaTeX}}
% \index{\texttt{\textbackslash xyz}@\hangleft{\texttt{\textbackslash}}\texttt{xyz}}
\newcommand{\tuftebs}{\symbol{'134}}% a backslash in tt type in OT1/T1
\newcommand{\doccmdnoindex}[2][]{\texttt{\tuftebs#2}}% command name -- adds backslash automatically (and doesn't add cmd to the index)
\newcommand{\doccmddef}[2][]{%
  \hlred{\texttt{\tuftebs#2}}\label{cmd:#2}%
  \ifthenelse{\isempty{#1}}%
    {% add the command to the index
      \index{#2 command@\protect\hangleft{\texttt{\tuftebs}}\texttt{#2}}% command name
    }%
    {% add the command and package to the index
      \index{#2 command@\protect\hangleft{\texttt{\tuftebs}}\texttt{#2} (\texttt{#1} package)}% command name
      \index{#1 package@\texttt{#1} package}\index{packages!#1@\texttt{#1}}% package name
    }%
}% command name -- adds backslash automatically
\newcommand{\doccmd}[2][]{%
  \texttt{\tuftebs#2}%
  \ifthenelse{\isempty{#1}}%
    {% add the command to the index
      \index{#2 command@\protect\hangleft{\texttt{\tuftebs}}\texttt{#2}}% command name
    }%
    {% add the command and package to the index
      \index{#2 command@\protect\hangleft{\texttt{\tuftebs}}\texttt{#2} (\texttt{#1} package)}% command name
      \index{#1 package@\texttt{#1} package}\index{packages!#1@\texttt{#1}}% package name
    }%
}% command name -- adds backslash automatically
\newcommand{\docopt}[1]{\ensuremath{\langle}\textrm{\textit{#1}}\ensuremath{\rangle}}% optional command argument
\newcommand{\docarg}[1]{\textrm{\textit{#1}}}% (required) command argument
\newenvironment{docspec}{\begin{quotation}\ttfamily\parskip0pt\parindent0pt\ignorespaces}{\end{quotation}}% command specification environment
\newcommand{\docenv}[1]{\texttt{#1}\index{#1 environment@\texttt{#1} environment}\index{environments!#1@\texttt{#1}}}% environment name
\newcommand{\docenvdef}[1]{\hlred{\texttt{#1}}\label{env:#1}\index{#1 environment@\texttt{#1} environment}\index{environments!#1@\texttt{#1}}}% environment name
\newcommand{\docpkg}[1]{\texttt{#1}\index{#1 package@\texttt{#1} package}\index{packages!#1@\texttt{#1}}}% package name
\newcommand{\doccls}[1]{\texttt{#1}}% document class name
\newcommand{\docclsopt}[1]{\texttt{#1}\index{#1 class option@\texttt{#1} class option}\index{class options!#1@\texttt{#1}}}% document class option name
\newcommand{\docclsoptdef}[1]{\hlred{\texttt{#1}}\label{clsopt:#1}\index{#1 class option@\texttt{#1} class option}\index{class options!#1@\texttt{#1}}}% document class option name defined
\newcommand{\docmsg}[2]{\bigskip\begin{fullwidth}\noindent\ttfamily#1\end{fullwidth}\medskip\par\noindent#2}
\newcommand{\docfilehook}[2]{\texttt{#1}\index{file hooks!#2}\index{#1@\texttt{#1}}}
\newcommand{\doccounter}[1]{\texttt{#1}\index{#1 counter@\texttt{#1} counter}}

% Generates the index
\usepackage{makeidx}
\makeindex

\begin{document}

\frontmatter

\maketitle

% v.2 epigraphs
\newpage\thispagestyle{empty}
\openepigraph{%
The public is more familiar with bad design than good design.
It is, in effect, conditioned to prefer bad design, 
because that is what it lives with. 
The new becomes threatening, the old reassuring.
}{Paul Rand%, {\itshape Design, Form, and Chaos}
}
\vfill
\openepigraph{%
A designer knows that he has achieved perfection 
not when there is nothing left to add, 
but when there is nothing left to take away.
}{Antoine de Saint-Exup\'{e}ry}
\vfill
\openepigraph{%
\ldots the designer of a new system must not only be the implementor and the first 
large-scale user; the designer should also write the first user manual\ldots 
If I had not participated fully in all these activities, 
literally hundreds of improvements would never have been made, 
because I would never have thought of them or perceived 
why they were important.
}{Donald E. Knuth}

% v.4 copyright page
\newpage
\begin{fullwidth}
~\vfill
\thispagestyle{empty}
\setlength{\parindent}{0pt}
\setlength{\parskip}{\baselineskip}
Copyright \copyright\ \the\year\ \thanklessauthor

\par\smallcaps{Published by \thanklesspublisher}

\par\smallcaps{tufte-latex.github.io/tufte-latex/}

\par Licensed under the Apache License, Version 2.0 (the ``License''); you may not
use this file except in compliance with the License. You may obtain a copy
of the License at \url{http://www.apache.org/licenses/LICENSE-2.0}. Unless
required by applicable law or agreed to in writing, software distributed
under the License is distributed on an \smallcaps{``AS IS'' BASIS, WITHOUT
WARRANTIES OR CONDITIONS OF ANY KIND}, either express or implied. See the
License for the specific language governing permissions and limitations
under the License.\index{license}

\par\textit{First printing, \monthyear}
\end{fullwidth}

% r.5 contents
\tableofcontents

% \listoffigures

% \listoftables

% r.9 introduction
%\cleardoublepage


%%
% Start the main matter (normal chapters)
\mainmatter



\chapter{Introduction and Goals} 

Hello! Allow me to introduce myself. My name is Niels, and by the time you read this, I will hopefully have a Ph.D from Stanford's Computer Science department. At the start of writing this document, I'm in my 6th year, and working towards a thesis focusing on automating the use of quadrotor aircraft equipped with cameras to perform cinematography! 

This collection of notes is my attempt to build a structure of sorts in which I can place the knowledge I'm gaining as I explore these worlds. You can view them as an account of my wanderings into the High Country of the Mind\cite{Pirsig2005}, exploring (mostly) old and (hopefully) a few new paths relating to the blossoming field of micro aerial vehicles and their usage in Computer Graphics.  
 \footnote{Shamelessly stealing metaphors from Robert Pirsig's Zen and the Art of Motorcycle Maintenance. One of my favorite books, both for a love of long distance motorcycle riding I share with the author, and it's views on the important strangeness of Quality.}

Now, before we jump into a big endeavor such as writing or reading a selection of notes, it's important that we establish some goals for this document. I find that, with good measurable goals publicly stated, I work much better and my colleagues and friends find it easier to collaborate with me. Hopefully the same effect will extend here!

With this document I'd like to:

\begin{enumerate}

\item Create a continuity of ideas over the multiple years of work and thought in this field, and thus avoid losing too much of what myself and my team learns and thinks to the rush of getting on to the Next Thing.

\item Learn through Writing. Writing concepts down, just like teaching, forces me to clarify them and patch holes in my understanding! \footnote{``Writing is nature's way of letting you know how sloppy your thinking is.'' - Guindon (cartoon)

``You have to learn, so you know what is it you know and don't know. In Science you must be very careful not to confuse yourself, which is very easy to do!'' - Richard Feynman.} 

\item Disseminate what I've learned to my colleagues, especially as interest in this area grows.

\item Track my own progress, what I don't know yet and what I've figured out, in a place where I can see the progress as I struggle and fight my way up into these mountains.\footnote{Keeping up the motivation as a graduate student by tracking progress is crucial!}

\item Record some of the things that didn't work, the trails that led to a dead end. There's no place to publish such stuff, yet 

\end{enumerate}

\chapter{Preface: Doing Research}

Research in Computer Science is a challening and potentially rewarding blend of:
\begin{itemize}
\item project management
\item reading and learning
\item thinking
\item problem-solving
\item experimentation
\item mathematical modeling
\item engineering systems
\item psychology studies
\item writing
\item public discourse in the form of public speaking and argueing
\item people management and collaboration skill
\end{itemize}
Becoming uncommonly good in all of these areas will serve you well during and after your Ph.D studies.

\section{Niels' Good Meeting Tips}

\begin{itemize}

\item[Agenda] A meeting needs an agenda. 

\end{itemize}

\section{Making Talks}
Talks are important, and fun! I refer you to the fantastic ``''

\section{Project Management Ideas}

\section{Maintaining Mental Health}

\subsection{Emotional Intelligence}

word up
indeed

\section{Research Traits to Develop}

- Integrity

- Courage



\chapter{Questions I'd Like To Know The Answer To}

There are many areas, both explored and new, that I'd like to understand thoroughly. It seems to me like the best way to do that is through asking questions. Questions are both the end-product and raw fuel of curiosity, so I'm yotting the questions I do not have a satisfying answer to here.

This section is truly intended for myself, and many a reader might find themselves either bored or stunned by my apparent ignorance. For that I apologize, I am in no way knowleadgable about everything I need to be, even less what I want to be. If you have an intuitive answer to these, please to email me a niels@cs.stanford.edu

\subsection{}

\subsection{Vision and Graphics}

\begin{enumerate}

\item How do we find correspondences between two sets of feature vectors?

\item How do we measure the difference and structure between two vector fields?

\end{enumerate}



\chapter{Dead Kittens Pile}






%%
% The back matter contains appendices, bibliographies, indices, glossaries, etc.

\backmatter

\bibliography{main}
\bibliographystyle{plainnat}


\printindex

\end{document}

